\documentclass{article}
% preamble.tex

% =============================================
% Package section
% =============================================

\usepackage[utf8]{inputenc}

% Margins
\usepackage{geometry}

% Fonts and generic style
\usepackage{helvet}
\usepackage{courier}
\usepackage[spanish,es-nodecimaldot]{babel}
\usepackage[utf8]{inputenc}
\usepackage[compact]{titlesec}
\usepackage{appendix}
\usepackage{lmodern}

% Paragraphs
\usepackage{parskip}
\usepackage{epigraph}
\usepackage{fancyhdr}

% Images
\usepackage{color}
\usepackage{graphicx}
\usepackage{xcolor}
\usepackage{lipsum}
\usepackage{epstopdf}
\usepackage{epsfig}
\usepackage{epsf}
\usepackage{float}
\usepackage{wrapfig}
\usepackage{lscape}
\usepackage{rotating}

% Links and referencies
\usepackage{booktabs}
\usepackage{xcolor}
\usepackage{subfigure}
\usepackage{mdframed}
\usepackage{hyperref}
\usepackage{csquotes}

% Figures
\usepackage[figurename=Ilustración]{caption}

% Math
\usepackage{showexpl}[rframe=none]
\usepackage[math]{blindtext}
\usepackage{amsmath}
\usepackage{pifont}

% Code
\usepackage{listings}
\usepackage{ifthen}

% Bibliografy
\usepackage[backend=biber]{biblatex}

% Tables
\usepackage{tabularx}
\usepackage{array}
\usepackage{multirow}

% Align
\usepackage{ragged2e}

% =============================================
% Style section
% =============================================

% Chapters
\setcounter{secnumdepth}{3}
\titleformat{\chapter}[display]
{\normalfont\huge\bfseries}{\chaptertitlename\ \thechapter}{20pt}{\Huge}

% Links and urls
\urlstyle{same}
\hypersetup{
	colorlinks   = false,
	linkcolor    = black,
	urlcolor     = blue,
	citecolor    = blue
}

% =============================================
% Command section
% =============================================

\renewcommand{\contentsname}{Contenido}
\renewcommand{\figurename}{Ilustración}
\renewcommand\labelitemi{\ding{52}}

% =============================================
% Definition section
% =============================================

\def\myauthorone{David Parreño Barbuzano}
\def\myauthortwo{Hector Miguel Martín Álvarez}
\def\mydegree{Grado en \\ Ingeniería Informática}
\def\myuniversity{Universidad \\ Las Palmas de Gran Canaria}
\def\mydate{mayo de 2022}
\def\mytitle{CamDoom - Propuesta de interacción multimodal percepto/efectora}
\def\mysubject{Creando Interfaces de Usuario}


\begin{document}

\begin{titlepage}
    \begin{center}
        {\bfseries\LARGE \myuniversity \par}
        \vspace{0.5cm}
        {\scshape\Large \mydegree \par}
        \vspace{3cm}
        {\scshape\Huge \mytitle \par}
        \vspace{2cm}
        \vfill
        {\Large Asignatura: \par}
        {\Large \mysubject \par}
        \vspace{2cm}
        {\Large Autores: \par}
        {\Large \myauthorone \par}
        \vspace{0.2cm}
        {\Large \myauthortwo \par}
        \vspace{0.5cm}
        \vspace{1cm}
    \end{center}
\end{titlepage}

\tableofcontents
\newpage

\section{Motivación/argumentación de la propuesta}

CamDoom es un videjuego de ordenador que es una versión muy reducida del clásico Doom
pero que utiliza la cámara web y la detección facial para la gestión del movimiento del
jugador. La motivación de este proyecto es la de desarrollar en Processing un videojuego
3D que utilice como controlador un dispositivo diferente al habitual teclado y ratón.

\section{Descripción técnica del trabajo realizado}

CamDoom utiliza las utilidades de Processing para la representación de modelos en 3D
que están definidos mediante ficheros OBJ, con figuras PShape, y también con imágenes
que simulan objetos tridimensionales. La implementación del videojuego utiliza un motor
primitivo que ofrece un sistema de colisiones, detección facial, animaciones, gestión
del sonido, entre otras características. Es importante recalcar que este motor no está
pensando como para ser fácilmente escalable, sino que fue diseñado con el objetivo de
ofrecer lo mínimo como para que CamDoom pueda ser jugable.

Respecto al movimiento, CamDoom utiliza la cámara web para gestionar el movimiento y
acciones del jugador. En esta listado se muestra cada uno de los controles:

\begin{itemize}
    \item Abrir la boca: moverse hacia adelante
    \item Mover hacia la derecha: girar hacia la derecha
    \item Mover hacia la izquierdo: girar hacia la izquierda
    \item Levantar las cejas: disparar si hay un enemigo detectado
\end{itemize}

\section{Fuentes y tecnologías utilizadas}

En este listado se muestran todas las tecnologías empleadas:

\begin{itemize}
    \item Sonidos del DOOM: \url{https://github.com/Olde-Skuul/doom3do}
    \item Fuente de texto del Título: \url{https://fontmeme.com/fonts/amazdoom-font/}
    \item Fuente de texto de las Opciones: \url{https://www.fontspace.com/category/doom}
    \item Librería Sounds de Processing: \url{https://github.com/processing/processing-sound}
    \item OSCP5: \url{https://github.com/sojamo/oscp5}
    \item QueasyCam: \url{https://github.com/jrc03c/queasycam}
    \item PeasyCam: \url{https://github.com/jdf/peasycam/}
    \item Guía para Detección de Figuras: \url{https://www.jeffreythompson.org/collision-detection/}
\end{itemize}

\section{Diario de actividades semanales}

El proyecto se llevó a cabo en una semana y en tres iteraciones. A continuación se explicará
con mayor detalle cada una de estas fases:

En la primera iteración se celebró una videollamada en la que los integrantes discutieron
acerca de la idea principal del proyecto, las características y restricciones del programa,
y las tareas a las que se iba a ocupar cada uno de ellos.

La segunda iteración consistió en otra videollamada en la que cada integrante comentó el
estado de sus tareas para así poder planificar y distribuir las siguientes tareas que se
encargaría cada integrante del proyecto.

Finalmente, en la tercera iteración todas las tareas ya estaban hechas y lo que se hizo
fue preparar los ficheros INSTALL, el README, y este mismo documento.

\section{Tareas realizadas y porcentaje de participación}

Al estar formando el grupo por dos personas, cada integrante se aplicó al completo en cada una de sus
tareas, contribuyendo en el desarrollo de la otra siempre que era necesario. En este listado se indica
cada una de las tareas realizadas por cada uno de los miembros.

\begin{table}[H]
    \centering
    \begin{tabular}{|l|l|}
        \hline
        \textbf{Héctor} & \textbf{David} \\
        \hline
        Búsqueda de librerías empleadas & Redacción de la memoria \\
        \hline
        Implementación del mapa & Control básico del GUI y del juego \\
        \hline
        Detección facial del jugador & Implementación del daño, salud, y escudo \\
        \hline
        Control del jugador con detección facial & Control con teclado y ratón \\
        \hline
        Localización de objetos y enemigos & Fichero AUTHORS e INSTALL \\
        \hline
        Implementación de la IA & Edición del CHANGELOG \\
        \hline
        Sistema de Colisiones & Búsqueda de imágenes, sonidos \\
        \hline
        Detección de enemigos & Implementación del mapa \\
        \hline
        Gestión del daño & Animaciones, sprites, fuentes \\
        \hline
        Edición del CHANGELOG & Gestión del Sonido \\
        \hline
        Edición del Vídeo Teaser & Diseño del GUI \\
        \hline
    \end{tabular}
\end{table}

\section{Enlace al código}

Repositorio Github: \url{https://github.com/HectorMartinAlvarez/CamDoom}
Vídeo Teaser: \url{https://github.com/HectorMartinAlvarez/CamDoom}

\section{Conclusiones y propuesta de ampliación}

CamDoom es un proyecto que tiene potencial pero que requiere de mucho
más tiempo como para ofrecer un producto con más contenido y mucho más
optimizado. Sin embargo, el videojuego proporciona una buena experiencia
de usuario con la que probar la interacción de un juego como Doom con los
gestos que se pueden hacer con la cara.

Respecto a la ampliación, se tiene pensado para un futuro que CamDoom
esté desarrollado con un motor que ofrezca más facilidades para escalar
e integrar nuevas mecánicas, animaciones, enemigos, entre otros. También
se quiere modificar las texturas originales del Doom para que el juego
tenga un aspecto que no se aleje del actual pero que sea diferente.

\section{Créditos materiales no originales del grupo}

Principalmente, el trabajo externo empleado para el diseño del juego
se obtuvo de los ficheros originales de la empresa id Software, Inc. aunque
el código fuente es totalmente original y no está basado en el motor de Doom.

\end{document}
